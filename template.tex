





\documentclass[10pt, a4paper, titlepage]{article}

\usepackage[T1]{fontenc}
\usepackage[utf8]{inputenc}
\usepackage{lmodern}
\usepackage{textcomp}
\usepackage{microtype}
\usepackage[english]{babel}
\usepackage{csquotes}
% Bibliographie -----------------------------------------
\usepackage[backend=biber,
			style=alphabetic,
			isbn=false,
			doi=false
			]{biblatex}
\addbibresource{./content/references.bib}

% Typographie -------------------------------------------
\renewcommand{\theenumi}{(\roman{enumi})}
\usepackage{graphicx}
\usepackage{setspace}
% -------------------------------------------------------

% Mathe Pakete, Einstellungen und Commands --------------
\usepackage{amsmath}
\usepackage{amsfonts}
\usepackage{amssymb}
\usepackage{amsthm}
\usepackage{euscript}
\numberwithin{equation}{section}


% Theorem-Umgebungen ------------------------------------
\usepackage{thmtools}

\declaretheorem[
	name=Theorem,
	numberwithin=section
	]{theorem}
\declaretheorem[
	name=Lemma,
	sibling=theorem,
	]{lemma}
\declaretheorem[
	name=Proposition,
	sibling=theorem,
	]{prop}
\declaretheorem[
	name=Corollary,
	sibling=theorem,
	]{corollary}

\declaretheorem[
	name=Definition,
	style=definition,
	numbered=no,
	]{definition}

\declaretheorem[
	name=Remark,
	style=remark,
	numbered=no
	]{remark}	
\declaretheorem[
	name=Example,
	style=definition,
	numbered=no
	]{example}


% Hyperref (immer am Schluss der Präambel!) -------------
\usepackage[pdftex,pdfpagelabels]{hyperref}
\hypersetup{pdftitle={Beweis des Satz von Thales},
			pdfauthor={Maxime Musterfrau}}
% -------------------------------------------------------

%%%%%%%%%%%%%%%%%%%%%%%%%%%%%%%%%%%%%%%%%%%%%%%%%%%%%%%%%
%%% Beginn des Dokuments
%%%%%%%%%%%%%%%%%%%%%%%%%%%%%%%%%%%%%%%%%%%%%%%%%%%%%%%%%
\begin{document}

% Titelei -----------------------------------------------
\hypersetup{pageanchor=false}
%!TEX root = ../thesis.tex
\begin{titlepage}
\vspace*{-2cm}  % bei Verwendung von vmargin.sty
\begin{flushright}
    \includegraphics[width=8cm]{./content/Uni_Logo_2016_SW.jpg}
\end{flushright}
%\vspace{1cm}

\begin{center}  % Diplomarbeit ODER Magisterarbeit ODER Dissertation
    \Huge{\textbf{\textsf{\MakeUppercase{
        Bachelorarbeit
    }}}}
    \vspace{2cm}

    \large{\textsf{  % Diplomarbeit ODER Magisterarbeit ODER Dissertation
                     % (Dies ist erst die Ueberschrift!)
        Titel der Bachelorarbeit
    }}
    \vspace{.1cm}

    \LARGE{\textsf{  Analytic sets
    }}
    \vspace{3cm}

    \large{\textsf{  % Verfasserin ODER Verfasser (Ueberschrift)
        Verfasser
    }}

    \Large{\textsf{  Charlotte Ahrendts
    }}
    \vspace{3cm}

    \large{\textsf{
        angestrebter akademischer Grad  % (Ueberschrift)
    }}

    \Large{\textsf{  % Magistra ODER Magister ODER Doktorin ODER Doktor
                     % ACHTUNG: Kuerzel "Mag.a" oder "Dr.in" nicht zulaessig
        Bachelor of Science (BSc.)
    }}
\end{center}
\vspace{2cm}

\noindent\textsf{Wien, im Monat September 2025}  % <<<<< ORT, MONAT UND JAHR EINTRAGEN
\vfill

\noindent\begin{tabular}{@{}ll}
\textsf{Studienkennzahl lt.\ Studienblatt:}
&
\textsf{UA 033621}  % <<<<< STUDIENKENNZAHL EINTRAGEN
\\
    % BEI DISSERTATIONEN:
%\textsf{Dissertationsgebiet lt. Studienblatt:}
    % ANSONSTEN:
\textsf{Studienrichtung lt.\ Studienblatt:}
&
\textsf{Mathematik}  % <<<<< DISSGEBIET/STUDIENRICHTUNG EINTRAGEN
\\
% Betreuerin ODER Betreuer:
\textsf{Betreuer: }
&
\textsf{Dipl.-Phys. Dipl.-Math. Dr. Peter Elbau}  % <<<<< NAME EINTRAGEN
\end{tabular}

\end{titlepage}


\newpage

\thispagestyle{empty}

%!TEX root = ../thesis.tex
\section*{Abstract}

Analytic sets form an often overlooked but nontheless very useful field of study. 
Their history teaches us about how mistakes lead to new discoveries and their current uses show how wanting to prove someone wrong can lead to important results in many different areas of mathematics. 

Analytic sets will be introduced as continuous images of Polish spaces. 
We will work towards an understanding of their nature, especially concerning differences and commonalities with Borel sets.

A main objective of this thesis will be a  proof of the fact that there exist Analytic sets which are not Borel. 
Further time will be spent on the measurability properties of Analytic sets and different frameworks in which they appear.



\newpage

%\thispagestyle{empty}
\tableofcontents

\newpage
% -------------------------------------------------------
\hypersetup{pageanchor=true}
\renewcommand{\thepage}{ \arabic{page} }

\setcounter{page}{1}
\onehalfspacing







\section{Introduction}
kurze History. Error von Lebesgue, Souslin/Luzin. Was bringen die kollegen uns\ldots

This thesis aims to give an overview over the topic of analytic sets. 
A field of study that came about in the early 20th century, born out of an error made by famous mathematician Henri Lebesgue. 
He remarked in a 1905 manuscript that the Borel sets are closed under projections. 
Only problem was that this is unfortunately not true. 
First to discover his error was the Russian mathematician Mikhail Y. Souslin in 1917. \cite{rogers1980}
\\
In proving Lebesgue wrong, Souslin along with his supervisor Luzin and the latters colleague Sierpiński, started to study the sets that do arise from the operations Lebesgue was performing.


Chapers 2,3 and 4 will closely follow the notation and proofs of \cite{cohn2013}


%%%%%%%%%%%%%%%%%%%%%%%%%%%%%%%%%%%%%%%%%%%%%%%%%%%%%%%%%%%%%

\section{Polish spaces and Analytic sets}
While we could in theory define analytic sets on a variety of spaces, the most common and by far most useful setting is that of polish spaces. These got their name in  honor of the polish mathematicians who were the first to extensively study them.



\begin{definition}[Polish space]
	A topological space $X$ is called a Polish space, if it is completely metrizable and seperable (contains a countable dense subset)
\end{definition}

Interesting to note here is the difference between a complete metric space and a completely metrizable space.
For the latter, we only require the existence of a complete metric on $X$, but we do not need to choose a concrete one.
This means of course that all complete metric spaces are Polish\\
\\
There are a few different ways to view analytic sets, as well as different ways to define them. 
Souslin, for example in his early works on the topic defined them as arising from a series of unions and intersections of certain families of sets.
We shall here however stick to the most common, and often most useful  definition, which is that of Analytic sets being continuous images of Polish spaces 

\begin{definition}[Analytic set]
	Let $X$ be a polish space, $A \subset X$. We call A analytic, if there exists a Polish space $Y$ and  $f:Y \to X$ continuous, such that  $f(Y) = A$

\end{definition}

Throughout this section, we will use some standard results about topological and metric spaces without proof. These are:
\begin{corollary}
	\label{second_countable}
	In metric spaces, seperability and second countablility are equivalent
\end{corollary}
\begin{corollary}
	\label{closed_cc}
	closed subsets of complete metric spaces are again complete metric spaces
\end{corollary}
Proofs for these theorems can be found in [cite/Anhang]

\begin{lemma}
	\label{products}
	Finite and countable products of Polish spaces are polish.
\end{lemma}
\begin{proof}
	Let $X_1,X_2,\ldots$ be a sequence of (nonempty) Polish spaces. We can choose a complete metric $\overline{d_i}$ for each of the spaces $X_i$. 
	Now for each $i$, let $d_i \left( x,y \right) := min \left\{ 1,\overline{d_i} \left( x,y \right)  \right\}$\\
	This again defines a complete metric with the additional property that $d_i\left( x,y \right) \leq 1$ for all $x,y \in X_i$ [Genauer beweisen? Anhang?]
This new metric retains only information about small distances in $\overline{d_i}$.\\
Now we can turn towards the cartesian product $X := \prod X_i$.
Let $d\left( x,y \right):= \sum \frac{1}{2^{n}} d_i\left( x_{i},y_{i} \right)$
where $x = \left( x_1,x_2,\ldots \right), y = \left( y_1,y_2,\ldots \right) \in X$ 
This sum converges for all $x,y \in X$ since we constructed the $d_i$ to be bounded by 1.
 Moreover, $d$ defines a metric on $X$. This is easily seen by the fact that positive definieteness, symmetry and the triangle inequality all hold in each term of our sum individually and thusfor the whole sum. 
 The topology generated by $d$ is exactly the product topology on $X$ [Genauer?].\\

[Ab hier nicht im Buch, hoffe der Beweis passt so] \\
To show that $d$ is indeed complete, we take an arbitrary Cauchy sequence $(x_{n})_{n \in  \mathbb{N}}$ in $X$. The projection onto each  space $X_i$ is also a cauchy sequence  $x_{i_n}$.
 Since $X_i$ is complete, $x_{i_n}$ converges to some value $x_i$. 
Let $x:= \left( x_1,x_2,\ldots \right) $ be the componentwise limit of our Cauchy sequences. We need to show that $x_{n}$ converges to $x$ in the metric $d$:\\
Choose an arbitrary $\varepsilon > 0 $ and $m \in  \mathbb{N}$, such that $\sum_{i \geq m} \frac{1}{2^i} \leq \frac{\varepsilon}{2} $.
For each $i \leq m$, we can find an  $N_i$, such that $d_i\left( x_{i_n},x_i \right) \leq \frac{\varepsilon}{2}$ for all $n\geq N_i$. 
If we choose $N := max_{i \leq m}\left\{ N_i \right\} $, we get the following esimate for $n \geq N$: 

\begin{align*}
	d\left( x_{n},x \right) &= \sum \frac{1}{2^i} d_i\left( x_{i_n},x_i \right)\\
				&= \sum_{i\leq m} \frac{1}{2^i} d_i\left( x_{i_n},x_i \right) + \sum_{i \geq m} \frac{1}{2^i} d_i\left( x_{i_n},x_i \right) \\
				&\leq \sum_{i \leq m} \frac{1}{2^i} \frac{\varepsilon}{2} + \sum_{i\geq m} \frac{1}{2^i} \\
				&\leq \frac{\varepsilon}{2} + \frac{\varepsilon}{2} = \varepsilon
\end{align*}
So we get convergence of $x_n$ and thus completeness of the metric space $\left( X,d \right) $ \\

What remains to be shown is seperability. We want to make use of the equivalence of seperability and second countability for metric spaces. [cite]
For each $i$, we can find a countable Basis  $\mathcal{U}_i$ for the topology on $X_i$.
The sets of the form $U_1 \times U_2 \times \ldots \times U_N \times X_{N+1} \times X_{N+2} \times \ldots$ form a countable Basis for the product topology on $X$, 
so  $X$ is a seperable metrisable space and thus Polish.

\end{proof}

\begin{corollary}
	Open and closed subsets of polish spaces are polish
\end{corollary}


\begin{proof}
	Let $X$ be a polish space, $A$ be an open or closed subset of $X$
	By Corollary \ref{second_countable}, $X$ has a countable basis of open sets for its Topology. Then the restrictions of those sets to  $A$ form  a countable basis for the subspace topology on $A$. 
By the same equivalence, we get seperability of  $A$.\\
If $A$ is closed, it is according to Corollary \ref{closed_cc} completely metrizable by any complete metric on $X$, resticted to  $A$.\\
Now only the case of  $A$ being an open subset of  $X$ remains.
Let  \[
d_0 \left( x,y \right) := d\left( x,y \right) + \left| \frac{1}{d\left( x,A^{C} \right) } - \frac{1}{d\left( y,A^C \right) } \right| 
.\]
where $d$ is a complete metric on  $X$ and  $d\left( x,A^C \right) := \inf\left( d\left( x,z \right): z \in A^C \right)  $.\\
$d_0$ defines a metric on $A$. [proof?]

\end{proof}


\begin{theorem}
	Open and closed subsets of Polish spaces are analytic
\end{theorem}

\begin{proof}
	Let $X$ be a Polish space, $A \subset X$ open or closed.
	We know that $A$, as a space is polish. So there exists a polish space $Y$ and a continuuous function  $f:Y\to A$, such that  $f(Y) = A$. 
	Let $\iota := A \hookrightarrow X$ be the canonical embedding of  $A$ into  $X$. 
	Then we can define  $\tilde{f} := f \circ \iota$. Then  $\tilde{f}:Y \to X $ and  $\tilde{f}(Y) = A$ hold. So  $A$ is an analytic subset of X


\end{proof}

Two particular spaces that are of great interest in the study of analytic sets are $\mathbb{N}^{\mathbb{N}}$ and $\left\{ 0,1 \right\}^{\mathbb{N}} $.
We will see that the polish space $Z$ in our definition of analytic set can always be replaced by the space $\mathbb{N}^{\mathbb{N}}$. But first we need to verify that they are in fact polish:

\begin{theorem}
	$\mathbb{N}^\mathbb{N}$ is polish
\end{theorem}
\begin{proof}
	by $\mathbb{N}$ we always mean the natural numbers together with the discrete topology in which every subset is open. Since $\mathbb{N}$ is countable, seperability immediately follows
	The discrete metric $d\left( m,n \right) = 1 - \delta_{mn} $, which equals is a complete metric on  $\mathbb{N}$.
	In this metric the only Cauchy sequences are those that are eventually constant which obviously converge.
	It now follows from \ref{products}, that the cartesian product $\mathbb{N}^{\mathbb{N}}$ is also Polish.
\end{proof}

\begin{theorem}
	$\left\{ 0,1 \right\}^\mathbb{N}$ is polish
\end{theorem}
\begin{proof}
	The proof for this is equivalent to that of $\mathbb{N}^{\mathbb{N}}$ being polish. We again choose the discrete topology and discrete metric on $\left\{ 0,1 \right\} $ and use \ref{products} to get that $\left\{ 0,1 \right\} ^{\mathbb{N}} $ is Polish.	
\end{proof}


\begin{theorem}
	Let $X$ be a Polish space. Then there is a continuous function $f: \mathbb{N}^\mathbb{N} \to X $, such that $f\left( \mathbb{N}^\mathbb{N} \right) = X$
\end{theorem}


\begin{theorem}
	Let $A$ be a nonempty analytic subset of a polish space $X$. Then there exists continuous function $f: \mathbb{N}^{\mathbb{N}} \to X$, such that $f\left( \mathbb{N}^{\mathbb{N}} \right) = A$
\end{theorem}


%%%%%%%%%%%%%%%%%%%%%%%%%%%%%%%%%%%%%%%%%%%%%%%%%%%%%%%%%%

\section{Borel-gedöns}
The sets maybe closest in nature to the analytic sets are the Borel sets. 
This of course is a natural consequence of the way analytic sets were discovered. 
In some sense, Analytic sets are a just a generalization of the Borel sets which are closed under projections. 
We will try to formalise this notion as well as some other results about these two classes of sets in this section.

\begin{definition}[Borel set]
	We call a subset of a topological space Borel, if it is a member of $\mathcal{B}$, the $\sigma-$Algebra generated by the open sets
\end{definition}


\begin{theorem}
	Let $B$ be a Borel subset of a Polish space X. Then B is analytic		
\end{theorem}

\begin{theorem}
	Let A be a subset of a polish space X. If $A$ and  $A^{C}$ are analytic, then $A$ ist Borel.  
\end{theorem}


\begin{definition}[Zero-Dimensional space?]
	.	
\end{definition}

\begin{theorem}
	Let B be a Borel subset of a Polish space X. Then there exists a zero-dimensional space Z, such that $f(Z) = B$
\end{theorem}

\begin{theorem}
	Let $B$ be an uncountable Borel subset of a polish space  $X$. Then  $B$ contains a subset which is homeomorphic to  $\left\{ 0,1 \right\}^\mathbb{N}$
	
\end{theorem}


\begin{theorem}[Separation theorem]
	.
\end{theorem}

\begin{theorem}
	There exists.
\end{theorem}

\begin{definition}[Borel isomorphic]
	We call two Borel subsets $A,B$ of a Polish space X Borel isomorphic, if there exists a bijective, Borel measurable function $f: A \to B$
\end{definition}


\begin{theorem}
	Two Borel subsets of a polish space X are Borel isomorphic iff they have the same cardinality
\end{theorem}

We have seen that all Borel sets are analytic, but have not yet said anything about the converse. So we turn to the foundational theorem, introduced by souslin, that makes this field of analytic sets worth studying:
\begin{theorem}[Souslin]
	There exist Analytic sets which are not Borel sets. 
\end{theorem}


\begin{example}
	As almost all 'nice' sets are Borel, we can assume that most constructions of Analytic non-borelian sets are fairly complicated. 
	One of the earliest such examples was provided in 1936 by polish mathematician Stefan Mazurkiewicz [cite]:\\
	Let $\mathbb{R}^I$ denote the space of real-valued functions, which are continuous in the closed interval $I:=[0,1]$. Let  $\Gamma$ be the set of functions  $f \in \mathbb{R}^I$, which are differentiable in $I$. This set is Co-analytic, meaning it is the complement of an analytic set, but is itself not analytic. 
	Thus, its complement cannot be Borel (or $\Gamma$ would be Borel as well, and thus Analytic, leading to a contradiction). \\
	The proof of this is rather long and technical and beyond the scope of this thesis, but can be found in [cite]
	
\end{example}

%%%%%%%%%%%%%%%%%%%%%%%%%%%%%%%%%%%%%%%%%%%%%%%%%%%%%%%%%%%%%%


\section{Measurability}

\begin{definition}[$\mu$-Measurable]
	
\end{definition}

\begin{definition}[Universally measurable]
	
\end{definition}

\begin{theorem}
	Every finite Borel measure on Polish space is regular
\end{theorem}

\begin{theorem}
	Let $B$ be an analytic subset of a polish space $X$. Then $B$ is universally measurable.
\end{theorem}

\begin{theorem}
	Let $(X,\mathcal{A})$,  $\left( Y, \mathcal{B} \right) $ be measurable spaces, that is, spaces endowed with a $\sigma$-Algebra.  
	Let $\mathcal{A}_*$ and $\mathcal{B}_*$ be the $\sigma$-Algebras of universally measurable sets.
	If  $f:X \to Y$ is  $\mathcal{A}- \mathcal{B}$-measurable, then it is also $\mathcal{A}_*-\mathcal{B}_*$-measurable
\end{theorem}


\begin{definition}[Analytic space]
	
\end{definition}

\begin{theorem}
	Let X,A be analytic meas. space, Y polish, f measurable then f(A) analytic.
\end{theorem}


%%%%%%%%%%%%%%%%%%%%%%%%%%%%%%%%%%%%%%%%%%%%%%%%%%%%%%%%%%%%%


\section{Alternative description}


\subsection{Souslin operation}
[Rogers, p.319 ?]

\subsection{Luzins construction ??}
[Sur les ensembles analytiques]


 \subsection{Projections}

%%%%%%%%%%%%%%%%%%%%%%%%%%%%%%%%%%%%%%%%%%%%%%%%%%%%%%%%%%%%%


\section{K-analytic sets??? prettyprettyplease?}
































\vspace{\fill}
\nocite{*}
\printbibliography{}
\end{document}
