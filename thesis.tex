\documentclass[12pt,a4paper]{article}
\usepackage[utf8]{inputenc}
\usepackage{amsmath}
\usepackage{amsfonts}
\usepackage{ mathtools , amssymb ,amsthm } 
\usepackage{graphicx}
\usepackage{pdfpages}
\usepackage{hyperref}
\usepackage{import}
\usepackage{xifthen}
\usepackage{pdfpages}
\usepackage{transparent}
\usepackage{ marvosym }
\usepackage{wrapfig}
\usepackage{tikz}
\usepackage{caption}
\usepackage{subcaption}


\newtheorem{theorem}{Theorem}[section]
\newtheorem{corollary}{Corollary}[section]
\newtheorem{lemma}[theorem]{Lemma}

\theoremstyle{remark}
\newtheorem*{remark}{Remark}

\theoremstyle{definition}
\newtheorem{definition}{Definition}[section]


\newcommand\blitz{\textup{\Lightning}}
\newcommand{\incfig}[1]{%
    \def\svgwidth{0.4\columnwidth}
    \import{./figures/}{#1.pdf_tex}
}


\title{Analytic sets \\ 
\large Bachelorarbeit betreut von Prof. Elbau}
\author{Charlie Ahrendts}

\begin{document}
\maketitle
\newpage

\section{Introduction}
kurze History. Error von Lebesgue, Souslin/Luzin. Was bringen die kollegen uns\ldots

This thesis aims to give an overview over the topic of analytic sets. 
A field of study that came about in the early 20th century, born out of an error made by famous mathematician Henri Lebesgue. 
He remarked in a 1905 manuscript that the Borel sets are closed under projections. 
Only problem was that this is unfortunately not the case. 
First to discover his error was the Russian mathematician Mikhail Y. Souslin in 1917.
In proving Lebesgue wrong, Souslin along with his supervisor Luzin and the latters colleague Sierpiński, started to study the sets that do arise from the operations Lebesgue was performing.





\section{Polish spaces and Analytic sets}
To study analytic sets, we first need a notion of Polish spaces since these form the framework in which Analytic sets operate. 
\begin{definition}[Polish space]
	A topological space $X$ is called a Polish space, if it is seperable, so contains a countable dense subset and completely metrizable.
\end{definition}

Interesting to note here is the difference between a complete metric space and a completely metrizable space.
For the latter, we only require the existence of a complete metric on $X$, but we do not need to choose a concrete one.
This means of course that all complete metric spaces are Polish\\
\\
There are a few different ways to view analytic sets, as well as different ways to define them. 
Souslin, for example in his early works on the topic defined them as arising from a series of unions and intersections of certain families of sets.
We shall here however stick to the most common, and often most useful  definition, which is that of Analytic sets being continuous images of Polish spaces 

\begin{definition}[Analytic set]
	Let $X$ be a polish space, $A \subset X$. We call A analytic, if there exists a Polish space $Y$ and  $f:Y \to X$ continuous, such that  $f(Y) = A$

\end{definition}


\begin{lemma}
	Finite and countable products of Polish spaces are polish.
\end{lemma}


\begin{theorem}[]
	$\mathbb{N}^\mathbb{N}$ is polish
\end{theorem}
\begin{proof}
	\ldots		
\end{proof}

\begin{theorem}[]
	$\left\{ 0,1 \right\}^\mathbb{N}$ is polish
\end{theorem}
\begin{proof}
	\ldots	
\end{proof}

\begin{theorem}[]
	Open and closed subsets of Polish spaces are analytic
\end{theorem}
\begin{proof}
	Let $X$ be a Polish space, $A \subset X$ open.
\end{proof}

\begin{theorem}[]
	Let $X$ be a Polish space. Then there is a continuous function $f: \mathbb{N}^\mathbb{N} \to X $, such that $f\left( \mathbb{N}^\mathbb{N} \right) = X$
\end{theorem}


\section{Borel-gedöns}

\section{Measurability}

%\begin{theorem}[]
%	Every finite Borel measure on Polish space is regular
%\end{theorem}


\section{K-analytic sets??? prettyprettyplease?}











\end{document}
